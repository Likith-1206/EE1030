\documentclass[journal]{IEEEtran}
\usepackage[a5paper, margin=10mm]{geometry}
%\usepackage{lmodern} % Ensure lmodern is loaded for pdflatex
\usepackage{tfrupee} % Include tfrupee package


\setlength{\headheight}{1cm} % Set the height of the header box
\setlength{\headsep}{0mm}     % Set the distance between the header box and the top of the text


%\usepackage[a5paper, top=10mm, bottom=10mm, left=10mm, right=10mm]{geometry}

%
\usepackage{gvv-book}
\usepackage{gvv}
\setlength{\intextsep}{10pt} % Space between text and floats

\makeindex

\begin{document}
\bibliographystyle{IEEEtran}
\onecolumn
\newpage
\title{2008-xe-'1-17'}
\author{AI24BTECH11004-Bheri Sai Likith Reddy}
\maketitle

\begin{enumerate}
       \item If the characteristic equation of a $3$x$3$ matrix is $\lambda^{3}-\lambda^{2}+\lambda-1=0$, then the matrix should be
	       \begin{enumerate}
             \begin{multicols}{4}  
		       \item Hermitian
		       \item unitary
		       \item skew symmetric
		       \item identity
             \end{multicols}
        	\end{enumerate}	
	\item $2 \lim _{(x, y) \rightarrow(0,0)} \frac{x^{4}+x y}{x^{3}-y^{3}}$ is
               \begin{enumerate}
			        \begin{multicols}{4}  
		       \item $0$
		       \item $1$
		       \item $-1$
		       \item does not exist
                    \end{multicols}   
	       \end{enumerate}	
       \item If $f(z)=u+i v$ is an analytic function and $u-v=(x-y)^{3}+k x y(x-y)$, then $k$ is
		\begin{enumerate}
			\item $2$
			\item $-4$
			\item $6$
			\item $-8$
		\end{enumerate}
	\item The directional derivative at the point $P(1,2,3)$ to the surface
$x^{2}+\frac{y^{2}}{4}+\frac{z^{2}}{9}=3$ in the direction of the vector $\overrightarrow{OP}$, where $O$ denotes the origin, is
		\begin{enumerate}
			\item $0$
			\item $\frac{2}{\sqrt{14}}$
			\item $\frac{3}{\sqrt{14}}$
	        \item $\frac{6}{\sqrt{14}}$
        	\end{enumerate}
	\item If the solution of the differential equation
          \begin{align*}
              \frac{d y}{d x}+P(x) y=x y^{3}
          \end{align*}
          is $y^{2}\brak{1+c e^{x^{2}}}=1$,
            $c$ being an arbitrary constant, then $P\brak{x}$ is
		\begin{enumerate}
             \begin{multicols}{4}  
		       \item $-x$
		       \item $\frac{x}{2}$
		       \item $x$
		       \item $2x$
             \end{multicols}
        	\end{enumerate}	
	\item The system of equations
            \begin{align*}
a x+b y+a^{2}=0 \\
b x+a y-b^{2}=0 \\
x+y+a-b=0
            \end{align*}
		\begin{enumerate}
			\item admits unique solution if $a=b \neq 0$
			\item admits unique solution if $a=-b \neq 0$
			\item admits unique solution if $a=b=0$
			\item does not admit unique solution
        	\end{enumerate}
	\item The matrix 
          \begin{align*}
              \myvec{l&0&\sin \theta \\ 0&1&m\\n&0&\cos \theta}
          \end{align*}
          is orthogonal, if
		\begin{enumerate}
              
		       \item $l=-\sin \theta, m=-\cos \theta, n=0$
		       \item $l=-\sin \theta, m=0, n=\cos \theta$
		       \item $l=\cos \theta, m=\sin \theta, n=0$
		       \item $l=-\cos \theta, m=0, n=\sin \theta$
            
        	\end{enumerate}	
	\item The radius of convergence of the real power series
             \begin{align*}
		\sum_{m=0}^{\infty} \frac{(m!)^{2}}{(2 m+1)!} x^{m} 
             \end{align*}
		is
		\begin{enumerate}
			\item $4$
            \item $3$
            \item $2$
            \item $1$
        	\end{enumerate}	
	\item  The value of
           \begin{align*}
		\brak{\int_{0}^{\frac{\pi}{2}}\brak{\sin \theta}^{3 / 4} d \theta} x\brak{\int_{0}^{\frac{\pi}{2}}(\sin \theta)^{-3 / 4} d \theta}
             \end{align*}
             is
		\begin{enumerate}
            \begin{multicols}{2}
			\item $\frac{2 \pi}{3}(\sqrt{2}+1)$
			\item $\frac{2 \pi}{3}(\sqrt{2}-1)$
			\item $\frac{\pi}{2} \sqrt{3}$
			\item $-\frac{\pi}{2} \sqrt{3}$
   \end{multicols}
        	\end{enumerate}	
	\item If $f\brak{z}=y\brak{1+x^{2}}+x^{2}+i\brak{y^{2}+2 y} x$ is differentiable at a point $z=z_{0}$, then $f'\brak{z_{0}}$ is
                \begin{enumerate}
			\item $0$
			\item $1$
			\item $i$
			\item $-i$
        	\end{enumerate}		
	\item The value of the integral
        \begin{align*}
            \oint_{\abs{z}=2} \frac{e^{1 / z}}{(z-1)^{2}} d z
        \end{align*}
        is
		\begin{enumerate}
			\item $0$
			\item $(2 e \pi) i$
			\item $(4 e \pi) i$
			\item $(4 \pi) i$
        	\end{enumerate}	
	\item The absolute value of the integral
           \begin{align*}
		\oint_{c}(-zdx+xdy+ydz),
             \end{align*}
             where $c$ is the curve obtained by the intersection of $x^{2}+y^{2}=a^{2}, a>0$ and $y=z$, is
		\begin{enumerate}
			\item $\frac{\pi a^{2}}{\sqrt{2}}$
			\item $\frac{\pi a^{2}}{\sqrt{3}}$
			\item $\pi a^{2} \sqrt{2}$
			\item $2 \pi a^{2}$
        	\end{enumerate}	
	\item One of the values of
           \begin{align*}
		\frac{1}{\brak{4 x^{2} D^{2}+8 x D+1}}(\ln x) \text { where } D \equiv \frac{d}{d x},
             \end{align*} 
             is
		\begin{enumerate}
			\item $\ln x+4$
			\item $\ln x-4$
			\item $4 \ln x-4$
			\item $4 \ln x+4$
        	\end{enumerate}	
	\item A particular integral of the differential equation
             \begin{align*}
		\frac{d^{2} y}{d x^{2}}-y=\sec h x
             \end{align*}
	 is
		\begin{enumerate}
			\item $-(\cosh x)(\ln \cosh x)+x \sinh x$
			\item $-(\sinh x)(\ln \cosh x)+x \cosh x$
			\item $(\cosh x)(\ln \sinh x)+x \sinh x$
			\item $(\sinh x)(\ln \cosh x)-x \cosh x$
        	\end{enumerate}
	\item If $u=u(x, t)$ is such that
                \begin{align*}
\frac{\partial^{2} u}{\partial t^{2}}=4 \frac{\partial^{2} u}{\partial x^{2}},
0 \leq x \leq \pi, t \geq 0,\\
u(0, t)=u(\pi, t)=0,
u(x, 0)=0,
\frac{\partial u}{\partial t}(x, 0)=\sin x,
                 \end{align*}
then $u\brak{\frac{\pi}{3}, \frac{\pi}{6}}$ is
		\begin{enumerate}
			\item $\frac{3}{4}$
			\item $\frac{3}{8}$
	         	\item $\frac{\sqrt{3}}{4}$
                 	\item $\frac{\sqrt{3}}{8}$
	\end{enumerate}	
    \item The two lines of regression of the variables $x$ and $y$ are $4 x+2.4 y=20$ and $1.6 x+4 y=12$.
The coefficient of correlation between $x$ and $y$ is
      \begin{enumerate}
			        \begin{multicols}{4}  
		       \item $0.49$
		       \item $-0.49$
		       \item $0.35$
		       \item $-0.35$
                    \end{multicols}   
	       \end{enumerate}
        \item  While solving the initial value problem
$\frac{d y}{d x}+k y=0, y(0)=1$ at $x=h$ by fourth order Runge-Kutta method, the expression for $k_{3}$ is
            \begin{enumerate}
			\item $-k h+\frac{(k h)^{2}}{2!}-\frac{(k h)^{3}}{3!}$
			\item $-k h+\frac{(k h)^{2}}{2}-\frac{(k h)^{3}}{3}$
			\item $-k h+\frac{(k h)^{2}}{2}-\frac{(k h)^{3}}{4}$
			\item $-k\brak{1+\frac{h}{2}-\frac{h^{2}}{3}}$
        	\end{enumerate}	
\end{enumerate}		
\end{document}

