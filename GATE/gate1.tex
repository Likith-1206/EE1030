\documentclass[journal]{IEEEtran}
\usepackage[a5paper, margin=10mm]{geometry}
%\usepackage{lmodern} % Ensure lmodern is loaded for pdflatex
\usepackage{tfrupee} % Include tfrupee package


\setlength{\headheight}{1cm} % Set the height of the header box
\setlength{\headsep}{0mm}     % Set the distance between the header box and the top of the text


%\usepackage[a5paper, top=10mm, bottom=10mm, left=10mm, right=10mm]{geometry}

%
\usepackage{gvv-book}
\usepackage{gvv}
\setlength{\intextsep}{10pt} % Space between text and floats

\makeindex

\begin{document}
\bibliographystyle{IEEEtran}
\onecolumn
\newpage
\title{2007-MA-'31-51'}
\author{AI24BTECH11004-Bheri Sai Likith Reddy}
\maketitle
\section{SECTION-A}
\begin{enumerate}
	\item Let $f(z)=2 z^{2}-1$. Then the maximum value of $\abs{f\brak{z}}$ on the unit disc $D=\cbrak{z \in C:\abs{z} \leq 1}$ equals
	       \begin{enumerate}
             \begin{multicols}{4}  
		       \item $1$
		       \item $2$
		       \item $3$
		       \item $4$
             \end{multicols}
        	\end{enumerate}	
	\item Let 
             \begin{align*}  
		f(z)=\frac{1}{z^{2}-3 z+2}
              \end{align*}
	         Then the coefficient of $\frac{1}{z^{3}}$ in the Laurent series expansion of $f(z)$ for $|z|>2$ is
               \begin{enumerate}
			        \begin{multicols}{4}  
		       \item $1$
		       \item $2$
		       \item $3$
		       \item $4$
                    \end{multicols}   
	       \end{enumerate}	
       \item Let $f: C \rightarrow C$ be an arbitrary analytic function satisfying $f\brak{0}=0$ and $f\brak{1}=2$. Then
		\begin{enumerate}
			\item there exists a sequence $\cbrak{z_{n}}$ such that $\abs{z_{n}}>n$ and $\abs{f\brak{z_{n}}}>n$
			\item there exists a sequence $\cbrak{z_{n}}$ such that $\abs{z_{n}}>n$ and $\abs{f\brak{z_{n}}}<n$
			\item there exists a bounded sequence $\cbrak{z_{n}}$ such that $\abs{f{z_{n}}}>n$
			\item there exists a sequence $\cbrak{z_{n}}$ such that $z_{n} \rightarrow 0$ and $f\brak{z_{n}} \rightarrow 2$
		\end{enumerate}
	\item Define $f: C \rightarrow C$ by 
         \begin{align*}
               f(x) = 
\begin{cases} 
    0 & \text{if Re \brak{z} or Im\brak{z}=0} \\
    z & otherwise.
\end{cases}
         \end{align*}
 Then the set of points where $f$ is analytic is
		\begin{enumerate}
			\item $\cbrak{z: Re \brak{z} \neq 0 \text{and} Im \brak{z} \neq 0}$
			\item $\cbrak{z: Re\brak{z} \neq 0}$
			\item $\cbrak{z: Re \brak{z} \neq 0 or Im \brak{z} \neq 0}$
	        \item $\cbrak{z: Im \brak{z} \neq 0}$
        	\end{enumerate}
	\item Let $U\brak{n}$ be the set of all positive integers less than $n$ and relatively prime to $n$. Then $U\brak{n}$ is a group under multiplication modulo $n$. For $n=248$, the number of elements in $U(n)$ is
		\begin{enumerate}
             \begin{multicols}{4}  
		       \item $60$
		       \item $120$
		       \item $180$
		       \item $240$
             \end{multicols}
        	\end{enumerate}	
	\item Let $ R \sbrak{x}$ be the polynomial ring in $x$ with real coefficients and let $I=\brak{ x^{2}+1}$ be the ideal generated by the polynomial $x^{2}+1$ in $R\sbrak{x}$. Then
		\begin{enumerate}
			\item $I$ is a maximal ideal
			\item $I$ is a prime ideal but NOT a maximal ideal
			\item $I$ is NOT a prime ideal
			\item $R\sbrak{x} / I$ has zero divisors
        	\end{enumerate}
	\item Consider $Z_5$ and $Z_{20}$ as rings modulo $5$ and $20 $, respectively. Then the number of homomorphisms $\phi : Z_5 \rightarrow  Z_{20}$ 
		\begin{enumerate}
             \begin{multicols}{4}  
		       \item $1$
		       \item $2$
		       \item $4$
		       \item $5$
             \end{multicols}
        	\end{enumerate}	
	\item Let $Q$ be the field of rational numbers and consider $Z_{2}$ as a field modulo $2$. Let
             \begin{align*}
		f(x)=x^{3}-9 x^{2}+9 x+3
             \end{align*}
		Then $f\brak{x}$ is
		\begin{enumerate}
			\item irreducible over $Q$ but reducible over $Z_{2}$
            \item irreducible over both $Q$ and $Z_{2}$
            \item reducible over $Q$ but irreducible over $Z_{2}$
            \item reducible over both $Q$ and $Z_{2}$
        	\end{enumerate}	
	\item  Consider $Z_{5}$ as a field modulo $5$ and let
           \begin{align*}
		f(x)=x^{5}+4 x^{4}+4 x^{3}+4 x^{2}+x+1
             \end{align*}
	     Then the zeros of $f\brak{x}$ over $Z_{5}$ are $1$ and $3$ with respective multiplicity
		\begin{enumerate}
            \begin{multicols}{2}
			\item $1$ and $4$
			\item $2 $ and $3$
			\item $2$ and $2$
			\item $1$ and $2$
   \end{multicols}
        	\end{enumerate}	
	\item Consider the Hilbert space $l^{2}=\cbrak{x=\cbrak{x_{n}}: x_{n} \in R, \sum_{n=1}^{\infty} x_{n}^{2}<\infty}$. Let
             \begin{align*}
		E=\cbrak{\cbrak{x_{n}}:\abs{x_{n}}\leq \frac{1}{n} \text { for all } n}
             \end{align*}
	     be a subset of $l^2$. Then
                \begin{enumerate}
			\item $E^{\circ}=\cbrak{x:\abs{x_{n}}<\frac{1}{n} \text{for all n}}$
			\item $E^{\circ}=E$
			\item $E^{\circ}=\cbrak{x:\abs{x_{n}}<\frac{1}{n} \text{for all but finitely many n}}$
			\item $E^{\circ}=\phi$
        	\end{enumerate}		
	\item Let $X$ and $Y$ be normed linear spaces and let $T: X \rightarrow Y$ be a linear map. Then $T$ is continuous if
		\begin{enumerate}
			\item $Y$ is finite dimensional
			\item $X$ is finite dimensional
			\item $T$ is one to one
			\item $T$ is onto
        	\end{enumerate}	
	\item Let $X$ be a normed linear space and let $E_{1}, E_{2} \subseteq X$. Define
           \begin{align*}
		E_{1}+E_{2}=\cbrak{x+y: x \in E_{1}, y \in E_{2}} 
             \end{align*}
             Then $E_{1}+E_{2}$ is
		\begin{enumerate}
			\item open if $E_{1}$ or $E_{2}$ is open
			\item NOT open unless both $E_{1}$ and $E_{2}$ are open
			\item closed if $E_{1}$ or $E_{2}$ is closed
			\item closed if both $E_{1}$ and $E_{2}$ are closed
        	\end{enumerate}	
	\item For each $a \in R$, consider the linear programming problem Max. $z=x_{1}+2 x_{2}+3 x_{3}+4 x_{4}$
subject to 
           \begin{align*}
		\begin{aligned}
& a x_{1}+2 x_{3} \leq 1 \\
& x_{1}+a x_{2}+3 x_{4} \leq 2 \\
& x_{1}, x_{2}, x_{3}, x_{4} \geq 0
\end{aligned} 
             \end{align*} 
	     Let $S=\cbrak{a \in R : \text{the given LP problem has a basic feasible solution} }$. Then
		\begin{enumerate}
			\item $S=\phi$
			\item $S=R$
			\item $S=(0, \infty)$
			\item $S=(-\infty, 0)$
        	\end{enumerate}	
	\item Consider the linear programming problem

		\begin{align*}
		Max. z=x_{1}+5 x_{2}+3 x_{3}
		\end{align*}
subject to
             \begin{align*}
		\begin{aligned}
& 2 x_{1}-3 x_{2}+5 x_{3} \leq 3 \\
& 3 x_{1}+2 x_{3} \leq 5 \\
& x_{1}, x_{2}, x_{3} \geq 0
\end{aligned}
             \end{align*}
	 Then the dual of this LP problem
		\begin{enumerate}
			\item has a feasible solution but does NOT have a basic feasible solution
			\item has a basic feasible solution
			\item has infinite number of feasible solutions
			\item has no feasible solution
        	\end{enumerate}
	\item Consider a transportation problem with two warehouses and two markets. The warehouse capacities are $a_{1}=2$ and $a_{2}=4$ and the market demands are $b_{1}=3$ and $b_{2}=3$. Let $x_{i j}$ be the quantity shipped from warehouse $i$ to market $j$ and $c_{i j}$ be the corresponding unit cost. Suppose that $c_{11}=1, c_{21}=1$ and $c_{22}=2$. Then $\brak{x_{11}, x_{12}, x_{21}, x_{22}}=(2,0,1,3)$ is optimal for every
		\begin{enumerate}
			\item $c_{12} \in[1,2]$
			\item $c_{12} \in[0,3]$
	         	\item $c_{12} \in[1,3]$
                 	\item $c_{12} \in[2,4]$
	\end{enumerate}	
    \item The smallest degree of the polynomial that interpolates the data
     \begin{tabular}{|l|l|l|l|l|l|l|}
     \hline$x$ & -2 & -1 & 0 & 1 & 2 & 3 \\
     \hline$f(x)$ & -58 & -21 & -12 & -13 & -6 & 27 \\
     \hline
\end{tabular}
is
      \begin{enumerate}
			        \begin{multicols}{4}  
		       \item $3$
		       \item $4$
		       \item $5$
		       \item $6$
                    \end{multicols}   
	       \end{enumerate}
        \item Suppose that $x_{0}$ is sufficiently close to $3$ . Which of the following iterations $x_{n+1}=g\brak{x_{n}}$ will converge to the fixed point $x=3$ ?
            \begin{enumerate}
			\item $x_{n+1}=-16+6 x_{n}+\frac{3}{x_{n}}$
			\item $x_{n+1}=\sqrt{3+2 x_{n}}$
			\item $x_{n+1}=\frac{3}{x_{n}-2}$
			\item $x_{n+1}=\frac{x_{n}^{2}-3}{2}$
        	\end{enumerate}	
\end{enumerate}		
\end{document}

