\documentclass[journal]{IEEEtran}
\usepackage[a5paper, margin=10mm]{geometry}
%\usepackage{lmodern} % Ensure lmodern is loaded for pdflatex
\usepackage{tfrupee} % Include tfrupee package


\setlength{\headheight}{1cm} % Set the height of the header box
\setlength{\headsep}{0mm}     % Set the distance between the header box and the top of the text


%\usepackage[a5paper, top=10mm, bottom=10mm, left=10mm, right=10mm]{geometry}

%
\usepackage{gvv-book}
\usepackage{gvv}
\setlength{\intextsep}{10pt} % Space between text and floats

\makeindex

\begin{document}
\bibliographystyle{IEEEtran}
\onecolumn
\newpage
\title{Assignment-2}
\author{AI24BTECH11004-Bheri Sai Likith Reddy}
\maketitle
 $\vec{P}\brak{5,-3}$ and $\vec{Q}\brak{3,y}$ are the points of trisection of the line segment joining $\vec{A}\brak{7,-2}$ and $\vec{B}\brak{1,-5}$. Theny equals\\
\solution Given $\vec{P}\brak{5,-3}$, $\vec{A}\brak{7,-2}$, $\vec{B}\brak{1,-5}$ and $\vec{Q}\brak{3,y}$\\
Also given that $\vec{P} and \vec{Q} $ are the points of tricection of $AB$.\\
Let q divides the line segment $AB$ in the ratio $k:1$.
That implies p divides line segment $AB$ in the ratio $1:k$.
\begin{align}
\implies p = \frac{ka+b}{k+1}\\
\implies \myvec{5\\ -3}=\frac{k\myvec{7\\ -2}+\myvec{1\\ -5}}{k+1}\\
\end{align}
lets solve x coordinate
\begin{align}
\implies 5=\frac{7k+1}{k+1}\\
\implies 5k+5=7k+1\\
\implies k=2
\end{align}
Therefore q divides $AB$ in the ratio $2:1$\\
\begin{align}
\implies \myvec{3\\ y}=\frac{b+\frac{1}{2}a}{1+\frac{1}{2}+1 }\\
\end{align}
lets solve y coordinate of q
\begin{align}
\implies y=\frac{\brak{-5}+\brak{-2}\frac{1}{2}}{\frac{3}{2}}\\
\implies y=\frac{-12}{3}
\end{align}
Therefore $y=-4$

\end{document}

