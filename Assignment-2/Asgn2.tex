\documentclass[journal]{IEEEtran}
\usepackage[a5paper, margin=10mm]{geometry}
%\usepackage{lmodern} % Ensure lmodern is loaded for pdflatex
\usepackage{tfrupee} % Include tfrupee package


\setlength{\headheight}{1cm} % Set the height of the header box
\setlength{\headsep}{0mm}     % Set the distance between the header box and the top of the text


%\usepackage[a5paper, top=10mm, bottom=10mm, left=10mm, right=10mm]{geometry}

%
\usepackage{gvv-book}
\usepackage{gvv}
\setlength{\intextsep}{10pt} % Space between text and floats

\makeindex

\begin{document}
\bibliographystyle{IEEEtran}
\onecolumn
\newpage
\title{Assignment-2}
\author{AI24BTECH11004-Bheri Sai Likith Reddy}
\maketitle
 $\vec{P}\brak{5,-3}$ and $\vec{Q}\brak{3,y}$ are the points of trisection of the line segment joining $\vec{A}\brak{7,-2}$ and $\vec{B}\brak{1,-5}$. Theny equals\\
\solution Given $\vec{P}\brak{5,-3}$, $\vec{A}\brak{7,-2}$, $\vec{B}\brak{1,-5}$ and $\vec{Q}\brak{3,y}$\\
Also given that $\vec{P} and \vec{Q} $ are the points of tricection of $AB$.\\
Let $\vec{Q}$ divides the line segment $AB$ in the ratio $k:1$.
That implies $\vec{P}$ divides line segment $AB$ in the ratio $1:k$.
\begin{align*}
	\vec{P}&= \frac{k\vec{A} +\vec{B}}{k+1}\\
\end{align*}
lets solve $x$ coordinate
\begin{align*}
	5=&\frac{7k+1}{k+1}\\
	k&=2
\end{align*}
Therefore $\vec{Q}$ divides $AB$ in the ratio $2:1$\\
\begin{align*}
	\myvec{3\\ y}&=\frac{\vec{B} +\frac{1}{2}\vec{A} }{1+\frac{1}{2}+1 }\\
\end{align*}
lets solve $y$ coordinate of $\vec{Q}$ 
\begin{align*}
	y&=\frac{\brak{-5}+\brak{-2}\frac{1}{2}}{\frac{3}{2}}\\
\end{align*}
Therefore $y=-4$
\begin{figure}[h]
    \centering
    \includegraphics[width=0.7\textwidth]{Fig/Figure_1.png}
\end{figure}
\end{document}

