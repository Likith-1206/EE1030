%iffalse
\let\negmedspace\undefined
\let\negthickspace\undefined
\documentclass[journal,12pt,twocolumn]{IEEEtran}
\usepackage{cite}
\usepackage{amsmath,amssymb,amsfonts,amsthm}
\usepackage{algorithmic}
\usepackage{graphicx}
\usepackage{textcomp}
\usepackage{xcolor}
\usepackage{txfonts}
\usepackage{listings}
\usepackage{enumitem}
\usepackage{mathtools}
\usepackage{gensymb}
\usepackage{comment}
\usepackage[breaklinks=true]{hyperref}
\usepackage{tkz-euclide} 
\usepackage{listings}
\usepackage{gvv}                                        
%\def\inputGnumericTable{}                                 
\usepackage[latin1]{inputenc}                                
\usepackage{color}                                            
\usepackage{array}                                            
\usepackage{longtable}                                       
\usepackage{calc} 
\usepackage{multirow}                                         
\usepackage{hhline}                                           
\usepackage{ifthen}                                           
\usepackage{lscape}
\usepackage{tabularx}
\usepackage{array}
\usepackage{float}


\newtheorem{theorem}{Theorem}[section]
\newtheorem{problem}{Problem}
\newtheorem{proposition}{Proposition}[section]
\newtheorem{lemma}{Lemma}[section]
\newtheorem{corollary}[theorem]{Corollary}
\newtheorem{example}{Example}[section]
\newtheorem{definition}[problem]{Definition}
\newcommand{\BEQA}{\begin{eqnarray}}
\newcommand{\EEQA}{\end{eqnarray}}
\newcommand{\define}{\stackrel{\triangle}{=}}
\theoremstyle{remark}
\newtheorem{rem}{Remark}

% Marks the beginning of the document
\begin{document}
\bibliographystyle{IEEEtran}

\title{JEE}
\author{AI24BTECH11004-BHERI SAI LIKITH REDDY}
\maketitle
\newpage
\bigskip

\renewcommand{\thefigure}{\theenumi}
\renewcommand{\thetable}{\theenumi}

\section{Section-A JEE Advanced/ IIT-JEE}
\begin{enumerate}
	\item $f\brak{x} = \mydet {
			\sec\brak{x} & \cos\brak{x} & \sec^2\brak{x} + \cot\brak{x} \cosec \brak{x}\\
		\cos^2\brak{x} & \cos^2\brak{x} & \cosec^2\brak{x}\\
		1 & \cos^2\brak{x} & \cos^2\brak{x}}$\\
	Then $\int_0^{\frac{\pi}{2}}$ $f\brak	{x}dx$ = \rule{1cm}{0.15mm}
\hfill{(1987 - 2 Marks)}\\ 
		
\item The integral $\int_0^{1.5} \brak{x^2}dx$,
\hfill{(1988 - 2 Marks)}\\
	Where \sbrak denotes the greatest integer finction, equals \rule{1cm}{0.15mm}  \\
		
\item The value of  $\int_{-2}^2 |1-x^2|dx$ is \rule{1cm}{0.15mm}
	\hfill{(1989 - 2 Marks)}\\
		
\item The value of $\int_{\frac{\pi}{4}}^{\frac{3\pi}{4}} \frac{\phi}{1+ \sin \phi} d\phi$
\hfill{(1993 - 2 Marks)}\\
		
\item The value of $\int_2^3 \frac{\sqrt x }{\sqrt {5-x}+\sqrt {x}} dx$
\hfill{(1994 - 2 Marks)}\\
		
\item If for nonzero $x$, $af\brak x+bf \left(\frac{1}{x} \right) = \frac{1}{x}-5 $ where $a$ 
	$\neq$  $b$, then $\int_1^2f(x)dx$= \rule{1cm}{0.15mm}
\hfill{(1996 - 1 Mark)}\\
		
\item If$ n>0$, $\int_1^{2\pi} \frac {x \sin ^{2n}x}{\sin^{2n}x+\cos^{2n} x} dx $
\hfill{(1996 -1 Mark)}\\
		
\item The value of $\int_1^{e^{37}} \frac {\pi \sin \brak{\pi lnx}}{x} dx$ is \rule{1cm}{0.15mm}

	\hfill{(1997 - 2 Marks)}\\
		
\item Let $\frac {d}{dx}F\brak{x}=\frac {e^{\sin\brak{x}}}{x},x>0$.If $\int_1^4 \frac{2e^{\sin \brak{x^2}}}{x}=F\brak{k}-F\brak{1}$ then one of the possible valued of$ k$ is \rule{1cm}{0.15mm}
\hfill{(1997 - 2 Marks)}\\
\end{enumerate}
\section{Section B True/False}
\begin{enumerate}
	\item The value of the intrgral $\int_0^{2a} \frac{f(x)}{\brak{f\brak{x}+f\brak{2a-x}}}dx$ is equal to $a$
\hfill{(1988 - 1 Mark)}\\
\end{enumerate}
\section{Section C MCQs with One Correct Answer}
\begin{enumerate}
	\item The value of the definite integral $\int_0^{2a}(1+e^{-x^2})dx$ is\\
		\begin{enumerate}
                	\item $-1$            
	                \item  $2$
	                \item  $1$+$e^{-1}$    
	\item  none of these
		\end{enumerate}
		\hfill{(1981 - 2 Marks)}
	\item Let $a,b,c$ be non-zero real numbers such that $\int_0^1 \brak{1+ \cos ^8 \brak{x}} \brak{ax^2 + bx +c} dx = \int_0^2 \brak{ 1+ \cos^8\brak{x} } \brak{ax^2+bx+c} dx.$\\
		Then the quadratic equation $ax^2+bx+c=0$ has
			\begin{enumerate}
			\item no roots in$ \brak{0,2}$
			\item at least one root in$ \brak{0,2}$
			\item double root in$ \brak{0,2}  $
	\item two imagenary roots
                        \end{enumerate}
		\hfill{(1981 - 2 Marks)}
	\item The area bounded by the curves $y=f\brak{x}$, the $x$-axis and the ordinate $x=1$ and $x=b$ is $\brak{b-1} \sin \brak{3b+4}$. Then $f\brak{x}$ is 
	                \begin{enumerate}
			\item$\brak{x-1} \cos \brak{3x+4}$
			\item$\sin {\brak{3x+4}}$
			\item$\sin {\brak{3x+4}+3\brak{x-1} \cos {\brak{3x+4}}}$
			\item none of the above
		        \end{enumerate}
		\hfill{(1982 - 2 Marks)}
	\item the value of the integral $\int_0^{\frac{\pi}{2}} \frac{\sqrt{\cot\brak{x}}}{\sqrt{\cot\brak{x}}+\sqrt{\tan\brak{x}}}dx$ is 
	                \begin{enumerate}
			\item $\frac{\pi}{4}$
		\item$\frac{\pi}{2}$
			\item$\pi$
		\item none of the above
		         \end{enumerate}
		\hfill{(1983 - 1 Marks)}
	\item For any integer $n$ the integral-\\
		$\int_0^{\pi} e^{\cos^2 \brak{x}} \cos ^3 {\brak{2n+1}}xdx$ has the value
                        \begin{enumerate}
			\item	$\pi$
			\item $1$
			\item $0$
			\item none of these
		        \end{enumerate}
		\hfill{(1985 - 2 Marks)}
\end{enumerate}
\end{document}
