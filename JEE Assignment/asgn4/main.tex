\documentclass[journal]{IEEEtran}
\usepackage[a5paper, margin=10mm]{geometry}
%\usepackage{lmodern} % Ensure lmodern is loaded for pdflatex
\usepackage{tfrupee} % Include tfrupee package


\setlength{\headheight}{1cm} % Set the height of the header box
\setlength{\headsep}{0mm}     % Set the distance between the header box and the top of the text


%\usepackage[a5paper, top=10mm, bottom=10mm, left=10mm, right=10mm]{geometry}

%
\usepackage{gvv-book}
\usepackage{gvv}
\setlength{\intextsep}{10pt} % Space between text and floats

\makeindex

\begin{document}
\bibliographystyle{IEEEtran}
\onecolumn
\newpage
\title{2023 February 1 Shift 2}
\author{AI24BTECH11004-Bheri Sai Likith Reddy}
\maketitle
\section{SECTION-A}
\begin{enumerate}
       \item The sum 
            \begin{align*}
	       \sum_{n=1}^\infty \frac{2n^2+3n+4}{(2n)!}
            \end{align*}
	is equal to :
	\hfill{[Jan 2023]}
	       \begin{enumerate}
		       \item $\frac{11e}{2}+\frac{7}{2e}$
		       \item $\frac{13e}{4}+\frac{5}{4e}-4$
		       \item $\frac{11e}{2}+\frac{7}{2e}-4$
		       \item $\frac{13e}{4}+\frac{5}{4e}$
        	\end{enumerate}	
	\item Let 
             \begin{align*}  
		S=\cbrak{x\in R: 0<x<1 and 2\tan ^{-1}\brak{\frac{1-x}{1+x}}=\cos ^{-1}\brak{\frac{1-x^2}{1+x^2}}}.
              \end{align*}
	 If $n\brak{S}$ denotes the number of elements in $S$ then :
	\hfill{[Jan 2023]}               
               \begin{enumerate}
			       \item $n\brak{S}=2$ and only one element in $S$ is less than $\frac{1}{2}$
                                \item $n\brak{S}=2$ and only one element in $S$ is less than $\frac{1}{2}$
				\item $n\brak{S}=2$ and only one element in $S$ is less than $\frac{1}{2}$ 
                                \item $n\brak{S}=0$
	       \end{enumerate}	
       \item Let $\overrightarrow{a}=2\hat{i}-7\hat{j}+5\hat{k}$, $\overrightarrow{b}=\hat{i}+\hat{k}$ and $\overrightarrow{c}=\hat{i}+2\hat{j}-3\hat{k}$ be three given vectors. If $\overrightarrow{r}$ is a vector such that $\overrightarrow{r}$x$\overrightarrow{a}=\overrightarrow{c}$x$\overrightarrow{a}$ and $\overrightarrow{r}.\overrightarrow{b}=0$, then $\abs{\overrightarrow{r}}$ is equal to :
       	\hfill{[Jan 2023]}
		\begin{enumerate}
			\item $\frac{11}{7}\sqrt{2}$
			\item $\frac{11}{7}$
			\item $\frac{11}{5}\sqrt{2}$
			\item $\frac{\sqrt{914}}{7}$
		\end{enumerate}
	\item If $A=\frac{1}{2}\myvec{1&\sqrt{3}\\-\sqrt{3}&1}$, then :
		\hfill{[Jan 2023]}
		\begin{enumerate}
			\item $A^{30}-A^{25}=2I$
			\item $A^{30}+A^{25}+A=I$
			\item $A^{30}+A^{25}-A=I$
	                \item $A^{30}=A^{25}$
        	\end{enumerate}
	\item Two sice are thrown independently. Let $A$ be the event that the number appeared on the $1^{st}$ die is less than the number appeared on the $2^{nd}$ die, $B$ be the event that the number appeared on the number appeared on the $1^{st}$ die is even and that on the second die is odd, and $C$ be the event that the number appeared on $i^{st}$ die is odd and that on the $2^{nd}$ is even. Then
		\hfill{[Jan 2023]}
		\begin{enumerate}
			\item the number of favourable cases of the event $\brak{A \cup B}\cap C$ is $6$
			\item $A$ and $B$ are mutually exchusive 
			\item The number of favourabel cases of the events $A,B$ and $C$ are $15,6$ and $6$ respectively 
			\item $B$ and $C$ are independent
        	\end{enumerate}	
	\item Which of the following statements is a tautology ?
		\hfill{[Jan 2023]}
		\begin{enumerate}
			\item $p \rightarrow \brak{p\wedge \brak{p \rightarrow q}}$
			\item $\brak{p\wedge q}\rightarrow \brak{\neg\brak{p}\rightarrow q}$
			\item $\brak{p\wedge \brak {p \rightarrow q}}\rightarrow \neg q$
			\item $p \lor \brak {p \wedge q}$
        	\end{enumerate}
	\item The number of integral values of $k$, for which one root of the equation 
              \begin{align*}
		x^2-8x+k=0
              \end{align*}
		      lies in the interval $(2,3)$, is:
		      	\hfill{[Jan 2023]}
		\begin{enumerate}
			\item $2$
			\item $0$
			\item $1$
			\item $3$
        	\end{enumerate}
	\item Let $f:R-{0,1} \rightarrow R$ be a function such that 
             \begin{align*}
		f\brak{x}+f\brak{\frac{1}{1-x}}=1+x.
             \end{align*}
		Then $f\brak{2}$ is equal to :
			\hfill{[Jan 2023]}
		\begin{enumerate}
			\item $\frac{9}{2}$
                        \item $\frac{9}{4}$
                        \item $\frac{7}{4}$
                        \item $\frac{7}{3}$
        	\end{enumerate}	
	\item  Let the plane $P$ pass through the intersection of the planes $2x+3y-z=2$ and $x+2y+3z=6$, and be perpendicular to the plan $2x+y-z+1=0$. If $d$ is the distance of $P$ from the point $\brak{-7,1,1}$, then $d_2$ is equal to:
		\hfill{[Jan 2023]}
		\begin{enumerate}
			\item $\frac{250}{83}$
			\item $\frac{15}{53}$
			\item $\frac{25}{83}$
			\item $\frac{250}{82}$
        	\end{enumerate}	
	\item Let $a, b$ be two real numbers such that $ab<0$. If the complex number $\frac{1+ai}{b+i}$ is of unit modulus and $a+ib $ lies on the circle $\abs{z-1}=\abs{2z}$, then a possible value of $\frac{1+\abs{a}}{4b}$, where \sbrak{t} is greatest inter function, is : 
	\hfill{[Jan 2023]}
                \begin{enumerate}
			\item $\frac{-1}{2}$
			\item $-1$
			\item $1$
			\item $\frac{1}{2}$
        	\end{enumerate}		
	\item The sum of the abosolute maximum and minimum values of the function 
             \begin{align*}
		f\brak{x} =\abs{x^2-5x+6}-3x+2
             \end{align*}
		in the interval $\sbrak{-1,3}$ is equal to:
			\hfill{[Jan 2023]}
		\begin{enumerate}
			\item $10$
			\item $12$
			\item $13$
			\item $24$
        	\end{enumerate}	
	\item Let $P\brak{S}$ denote the power set of $S =\cbrak{1,2,3,\ldots,10}$. Define the relations $R_1$ and $R_2$ on $P\brak{S}$ as $AR_1B$ if $\brak {A \cap B^c}\cup \brak{B\cap A^c}=\phi$ adn $AR_2B$ if $A\cup B^c=B\cup A^c$, $\forall$ A,B $\in$ P\brak{S}. Then :
		\hfill{[Jan 2023]}
		\begin{enumerate}
			\item both $R_1$ and $R_2$ are equivalence relations
			\item only $R_1$ is an equivalence realtion
			\item only $R_2$ is an euevalence realtaion
			\item both $R_1$ and $R_2$ are not equivalence relations
        	\end{enumerate}	
	\item The area of the region given by $\cbrak{\brak{x,y}:xy \leq 8, 1 \leq y \leq x^2 } is :$
		\hfill{[Jan 2023]}
		\begin{enumerate}
			\item $8 \log^{2}_e-\frac{13}{3}$
			\item $16 \log^{2}_e-\frac{14}{3}$
			\item $8 \log^{2}_e+\frac{7}{6}$
			\item $16 \log^{2}_e+\frac{7}{3}$
        	\end{enumerate}	
	\item Let $\alpha x=exp\brak{x^\beta y^\gamma}$ be the solution of the differential equation 
             \begin{align*}
		2x^2ydy-\brak{1-xy^2}dx=0,
             \end{align*}
	 $x>0$, $y\brak{2}$=$\sqrt{\log_e^2}$. Then $\alpha + \beta - \gamma$ equals :
	 	\hfill{[Jan 2023]}
		\begin{enumerate}
			\item $1$
			\item $-1$
			\item $0$
			\item $3$
        	\end{enumerate}
	\item The value of the integral 
             \begin{align*}
		\int_\frac{-\pi}{4}^\frac{\pi}{4}\frac{x+\frac{\pi}{4}}{2-\cos 2x}dx
             \end{align*}
	     is :
	     	\hfill{[Jan 2023]}
		\begin{enumerate}
			\item $\frac{\pi^2}{6}$
			\item $\frac{\pi^2}{12\sqrt{3}}$
	         	\item $\frac{\pi^2}{3\sqrt{3}}$
                 	\item $\frac{\pi^2}{6\sqrt{3}}$
	\end{enumerate}	
\end{enumerate}		
\end{document}

